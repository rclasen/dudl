\documentclass[a4paper,10pt]{article}

\usepackage{german}
\usepackage[latin1]{inputenc}


\begin{document}

\section{�berblick}

\subsection{Datenbank}

die Datenbank soll enthalten:
\begin{description}
\item[User]
	Wer hat welchen login und Passwort. Welche Rechte geh�ren zu einem
	login.
\item[Musik]
	Welche Titel geh�ren zu welchem Album eines K�nstlers.
\item[Rating]
	In welches Genre sortiert ein User einen Titel. Wie gerne h�rt er
	einen Titel. Zusammenstellung von Programmen.
\item[Storage]
	Auf Welchem Medium (CD, MC, MP3, \ldots) ist ein Titel oder Album
	zu finden - bzw. Welche Alben sind auf einem Medium.
\item[Player]
	Ausw�hlen und abspielen von tracks.
\item[Video]
	Was ist auf Video Kassetten.
\item[Verleih]
	Wer hat welches Video oder Musik Tr�ger wann ausgeliehen.
\end{description}

Die Datenbank soll mindestens �ber ein Web Interface abfrag- und
editierbar sein.

\subsection{Player}

Der Player soll Modular aufgebaut sein. Die einzelnen Teile m�ssen direkt
miteinander kommunizieren. Im Idealfall greift nur ein Teil auf die
Datenbank zu und stellt allen anderen komponeneten des Players die Daten
zur verf�gung. Zu Begin ist aber denkbar, da� die anderen Komponenten noch
lesend zugriff auf die User-, Musik- und Storage Datenbank erhalten. Der
Player Spezifische Teil sollte auch schon zu Anfang direkt ausgetauscht
werden um ereignisgesteuert vorgehen zu k�nnen.

Der Player wird in folgende Einheiten zerteilt:

\begin{description}
\item[Frontend]
	Ein Frontend ist f�r das User interface zust�ndig. Es kann
	�ber ein eigenes Protokoll mit dem Backend kommunizieren.

\item[Player]
	Der eigentlich Player ist ein Teil des Backends. Er k�mmert sich
	dadrum, den richtigen Track zu spielen. Er reagiert auf die
	frontends und nimmt sich die n�tigen Daten aus der Datenbank. Der
	Player bereitet Die Daten f�r den Playback Client auf. Nur ein
	einziger Playback Client kann mit einem Player verbunden sein. Es
	sollen mehrere Player laufen k�nnen.

\item[Playback Client]
	Verbindet sich zu einem Player des Backends. Er nimmt die Daten
	in allen f�r ihn verwertbaren formaten entgegen und sorgt daf�r,
	da� sie abgespielt werden k�nnen - zB. auf /dev/dsp, �ber ESD oder
	�ber einen schoutcast server.

	Playback client und Player handeln zu begin aus, welche Formate
	vom client unterst�tzt werden.

	Sp�ter sollen auch die Musikdaten vom Player zum Playback Client
	geschickt werden. Im Anfangsstadium ist auch direkter Access via
	NFS + AMD denkbar.

\item[Backend]
	Das Backend kapselt die Datanbank. Frontends und Playback Clients
	verbinden sich mit dem Backend und erhalten �ber das Backend
	Daten. Im Idealfall ist das Backend der einzige mit zugriff auf
	die Datenbank.

\end{description}



\subsubsection{Frontends}

geplant sind Frontends f�r:

\begin{itemize}
\item cmdline/readline
\item LCDproc (und LIRC)
\item WWW
\item GTK
\item curses/dialog
\item Windows *w�rg*
\end{itemize}





\section{Die Musikdatenbank}
\begin{itemize}
\item neues Album
\item track(s) an Album anf�gen
\item wer hat wann in welcher Band gespielt
\item Suche von
	\begin{itemize}
	\item Interpret
	\item track
	\item album
	\end{itemize}
nach
	\begin{itemize}
	\item track
	\item Interpret
	\item Album
	\item Release Datum
	\item publisher
	\end{itemize}
\end{itemize}




\section{Die Storage Datenbank}

\begin{itemize}
\item welche tracks/alben sind auf welchem Medium
\item Wann aufgenommen
\item von wo aufgnommen
\item aufnahmequalit�t
\item wann gekauft
\item meine eigene CD - sprich in meinem Schrank, oder nur mal geliehen
gewesen und f�r CDDB/kopieren eingelesen?
\item f�r H�llen (Kassette) Inlays erzeugen
\item Listen mit alben erzeugen
\item vorkommen eines/des besten Titels/Albums aus Musikdb suchen (auf bestimmten
medien)
\item alternativen zu einem Track auf anderem Medium finden
\item erweiterbar f�r neue Medien
\item suche von CD Text nach CDDB ID - ggf. export an CDDB Daemon
\item suche nach CDROM anhand von Volume ID - ggf. export an vold oder so
\item automatisches einlesen (cddb) von CDs
\item automatisches scannen von CDROMs nach bekannten filetypen
\item sonst bei neuen medien auswahl aus musikdb anbieten - am besten
eines albums, das dann automatisch �bernommen wird.
\item ist ein File mit gegebener MD5 Summe schon in Datenbank?
\end{itemize}




\section{Video}
\begin{itemize}
\item Welche Filme sind auf welcher Kassette
\item Wer hat bei dem film als was mitgewirkt
\item Welche Musiktitel sind abgespielt worden
\item Wann aufgenommen
\item von wo aufgnommen
\item aufnahmequalit�t
\end{itemize}




\section{User}
\begin{itemize}
\item Wer hat mit welchem login welches passwort?
\item Zu welchen Gruppen geh�rt er?
\end{itemize}




\section{Verleih}
\begin{itemize}
\item verleih aller Medien an oben stehende Medien (Musik, Video)
\item Wer hatte wan was ausgeliehen
\end{itemize}




\section{Rating}
\begin{itemize}
\item zu welchem genre geh�rt ein track?
\item wie gef�llt mir der track
\item wie gef�llt mir ein genre?
\item zusammenstellen von programmen aus storage
\item Programme aufbrechen in happen bestimmter Spielzeit
\item Programm f�r Medium speichern
\end{itemize}





\section{Der Player}

\begin{itemize}
\item zugriff je nach Benutzer regeln
\item tracks aus einer queue spielen
\item queue aus programmen f�llen
\item Trackweise in queue vor- und in history r�ckw�rts springen
\item Random play gem�� filter und rating der lauschenden user/History
\end{itemize}

\subsection{Zusammenspiel queue und History}

Diese Spr�nge in der queue\ldots
\\
\begin{tabular}{r|ccccccc}
queue	& 	&	&	&	&	&	&	\\
\hline
10	& 1	&	& 	& 	&	&	& 	\\
11	& 2	& 6	& 6	& 	&	&	& 	\\
12	& 3	& 5	& 7	& 	&	& 15	& 15	\\
13	& 4	& 4	& 8	& 12	& 12	& 14	& 16	\\
14	&	&	& 9	& 11	& 13	& 13	& 17	\\
15	&	&	& 10	& 10	&	&	& 18	\\
\end{tabular}
\\
\ldots ergeben diese history/queue
\\
\begin{tabular}{rrr|l}
history	&	&	& queue \\
id	& prev	& track & tracks \\
\hline
1	& -	& 10	& 11, 12, 13, 14, 15 \\
2	& 1	& 11	& 12, 13, 14, 15 \\
3	& 2	& 12	& 13, 14, 15 \\
4	& 3	& 13	& 14, 15 \\
\hline
5	& 2	& 12	& 13, 14, 15 \\
6	& 1	& 11	& 12, 13, 14, 15 \\
\hline
7	& 6	& 12	& 13, 14, 15 \\
8	& 7	& 13	& 14, 15 \\
9	& 8	& 14	& 15 \\
10	& 9	& 15	& \\
\hline
11	& 8	& 14	& 15 \\
12	& 7	& 13	& 14, 15 \\
\hline
13	& 12	& 14	& 15 \\
\hline
14	& 6	& 13	& 14, 15 \\
15	& 1	& 12	& 13, 14, 15 \\
\hline
16	& 15	& 13	& 14, 15 \\
17	& 16	& 14	& 15 \\
18	& 17	& 15	& \\
\end{tabular}

\end{document}
